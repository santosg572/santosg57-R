\begin{frame}{Frame Title}
    \frametitle{Conjuntos}

Hay otras maneras de combinar conjuntos para obtener otros conjuntos.

\hfill

El \textbf{producto cartesiano} de dos conjuntos A y B es el conjunto
$A \times B$ cuyos elementos son todas las parejas ordenadas (a,b)
formadas por un elemento de A y un elemento de B:
$$
A \times B = \{ (x,y) | x \in  A , y \in B \}
$$
Ejemplos.

\hfill

\begin{itemize}
   \item Si $A = \{ 1,2,3 \}$ y $B = \{ 1,2 \}$ entonces $A \times B = \{ (1,1), (1,2), (2,1), (2,2), (3,1), (3,2) \}$
   \item Si $A = \{1,2,3 \}$ y $B = \{ \}$ entonces $A \times B = \{ \}$
   \item Si $\mathbb{R}$ es el conjunto de los números reales, entonces $\mathbb{R} \times \mathbb{R}$ es el conjunto de parejas ordenadas de
números reales, también denotado por $\mathbb{R}^2$
\end{itemize}

En el producto cartesiano el orden importa: si $A \neq B$ entonces $A \times B \neq  B \times A$.

\end{frame}

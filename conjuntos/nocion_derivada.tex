\begin{frame}{Frame Title}
    \frametitle{Noción de derivada}

Las derivadas se definen tomando el límite de la pendiente de las rectas secantes conforme se van aproximando a la recta tangente. Es difícil hallar directamente la pendiente de la recta tangente de una función porque sólo se conoce un punto de esta, el punto donde ha de ser tangente a la función. Por ello, se aproxima la recta tangente por rectas secantes. Cuando se tome el límite de las pendientes de las secantes próximas, se obtendrá la pendiente de la recta tangente. 


\end{frame}

\begin{frame}{Frame Title}
    \frametitle{Conjuntos}

Para obtener estas pendientes, tómese un número arbitrariamente pequeño que se denominará h. 
h representa una pequeña variación en x, y puede ser tanto positivo como negativo. 
La pendiente de la recta entre los puntos $( x , f ( x ) )$ y $( x + h , f ( x + h ) )$  es

$$
\frac{(x+h)-f(x)}{h}
$$

Esta expresión es un cociente diferencial de Newton. La derivada de f en x es el límite del valor del cociente diferencial conforme las líneas secantes se acercan más a la tangente:

$$
f'(x)=\lim_{h \to 0} \frac{f(x+h)-f(x)}{h}
$$

Si la derivada de f existe en cada punto x, es posible entonces definir la derivada de f como la función 
cuyo valor en el punto x es la derivada de f en x. 

\end{frame}

\begin{frame}{Frame Title}
    \frametitle{Conjuntos}

Puesto que la inmediata sustitución de h por 0 da como resultado una división por cero, calcular la derivada directamente puede ser poco intuitivo. Una técnica consiste en simplificar el numerador de modo que la h del denominador pueda cancelarse. Esto resulta muy sencillo con funciones polinómicas, pero para la mayoría de las funciones resulta demasiado complicado. Afortunadamente, hay reglas generales que facilitan la diferenciación de la mayoría de las funciones. 

\end{frame}

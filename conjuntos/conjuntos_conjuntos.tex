\begin{frame}{Frame Title}
    \frametitle{Conjuntos de conjuntos}

\textbf{Conjuntos de conjuntos.}

Podemos considerar a conjuntos cuyos elementos sean otros conjuntos.

\hfill

\textbf{Ejemplos.}

\hfill

\begin{itemize}
   \item El conjunto formado por los grupos en una escuela.
   \item El conjunto de todos los intervalos en la recta.
   \item El conjunto de todas las rectas en el plano (este conjunto no es el plano, que es un conjunto de
puntos, sino un conjunto cuyos elementos son las rectas)
\end{itemize}

Si A es cualquier conjunto, el \textbf{conjunto potencia} de A, es el conjunto cuyos elementos son todos
los subconjuntos de A, y es denotado por $2^A$.

\end{frame}

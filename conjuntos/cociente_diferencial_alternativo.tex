\begin{frame}{Frame Title}
    \frametitle{El cociente diferencial alternativo}

\textbf{El cociente diferencial alternativo}

La derivada de f(x) (tal como la definió Newton) se describió como el límite, conforme h se aproxima a cero. 
Una explicación alternativa de la derivada puede interpretarse a partir del cociente de Newton. 
Si se utiliza la fórmula anterior, la derivada en c es igual al límite conforme h se aproxima a cero de [f(c + h) - f(c)] / h. 
Si se deja que h = x - c (por ende, c + h = x), entonces x se aproxima a c (conforme h tiende a cero). 
Así, la derivada es igual al límite conforme x se aproxima a c, de [f(x) - f(c)] / (x - c). 
Esta definición se utiliza para una demostración parcial de la regla de la cadena. 


\end{frame} 

\begin{frame}{Frame Title}
    \frametitle{Conjuntos}


\end{frame} 

\begin{frame}{Frame Title}
    \frametitle{Conjuntos}


\end{frame} 

\begin{frame}{Frame Title}
    \frametitle{Conjuntos}


\end{frame} 
\end{document}

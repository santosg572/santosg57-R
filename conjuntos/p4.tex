\begin{frame}{Frame Title}
    \frametitle{Conjuntos}

Un \textbf{conjunto vacío} es un conjunto que no tiene elementos, así que todos los conjuntos vacíos
son iguales, lo denotamos por $\{ \}$ o por $\O$.

\hfill

\textbf{Ejemplos.}

\hfill

\begin{itemize}
\item El conjunto de todos los perros voladores = $\O$ = El conjunto de los triángulos con 4 lados.
\item $\O = \{ \} \neq \{\{ \}\} = \{ \O \}$
\end{itemize}

\hfill

Si A y B son conjuntos, decimos que A esta \textbf{contenido} en B, o que A es \textbf{subconjunto} de B, si
todos los elementos de A son elementos de B, y escribimos $A \subset B$.
$$
A \subset B \ \forall \ x, x \in A \Rightarrow  x \in B
$$

Por el contrario, cuando A no esta contenido en B escribimos $A \nsubseteq B$

$$
A \nsubseteq B \ \forall \ \exists \ x, x \in A \notin B
$$

\end{frame}


\begin{frame}{Frame Title}
    \frametitle{Conjuntos universales y complementos}

\textbf{Conjuntos universales, complementos.}

\hfill

Si fijamos un conjunto universal U y A está contenido
en U, el \textbf{complemento} de A (en U) es el conjunto $A^c$
formado por los elementos de U que no están en A.

$$
A^c = \{x \in U | x \notin  A \} = U - A
$$

\textbf{Ejemplos.}

\begin{itemize}
   \item Si pensamos en el universo A de todos los animales y el conjunto de los animales que vuelan es V
entonces su complemento $V^c$ es el conjunto de los animales que no vuelan.
   \item La propiedad que define al complemento $A^c$ es la negación de la propiedad que define a A, pero $A^c$
depende de cual sea el universo.
\end{itemize}
\end{frame}

\begin{frame}{Frame Title}
    \frametitle{Conjuntos universales y complementos}

\begin{itemize}
   \item Si $\mathbb{P} = \{ n \in \mathbb{N} | n \text{ es primo } \}$ entonces se entiende que el conjunto 
universal es $\mathbb{N}$ (los números
naturales) y su complemento es $\mathbb{P}^c = \{n \in \mathbb{N} | n \text{no es primo} \}$
%   \item El conjunto de los números racionales es el conjunto
%$ \mathbb{Q} = { x \in \mathbb{R} | \text{ existen enteros m y n tales que x =m/n} \}$
%   \item Su complemento (en el universo de los números reales) es el conjunto de los números irracionales:
%   $\mathbb{I} = \{ x \in \mathbb{R} | \text{ no existen enteros m y n tales que x =m/n} \}$
\end{itemize}

\end{frame}

\begin{frame}{Frame Title}
    \frametitle{Conjuntos universales y complementos}

\textbf{Lema.} Si A y B son conjuntos en un universo U, entonces $A - B = A \cap B^c$

\hfill

\textbf{Lema.} Si A y B son conjuntos en un universo U, entonces $A \subset B$ si y solo si $B^c \subset A^c$


\end{frame}

\documentclass{beamer}
\usetheme{Madrid} % Optional theme

\title{Teoría de Conjuntos}
\author{L. González-Santos}
\date{\today}

\begin{document}

% Title Page
\begin{frame}
    \titlepage
\end{frame}

% Table of Contents
%\begin{frame}{Outline}
%    \tableofcontents
%\end{frame}

% Example Content Frame
\section{Introduction}

\begin{frame}{Frame Title}
    \frametitle{Conjuntos}

Intuitivamente, un \textbf{conjunto} es una colección de objetos, reales o imaginarios, llamados
\textbf{elementos} del conjunto. Esta definición tiene algunos problemas cuando los conjuntos son muy
grandes, pero lo importante es que cada conjunto está determinado por sus elementos:

\hfill

Dos conjuntos son iguales si tienen los mismos elementos.

\hfill

Para decir que x es elemento de A escribimos $x \in A$, y al conjunto que contiene a x,y,z se le
denota por $\{ x,y,z \}$ donde el orden no importa y no hay repeticiones.

\end{frame}

\begin{frame}{Frame Title}
    \frametitle{Conjuntos}

\textbf{Ejemplos.}

\begin{itemize}
\item El conjunto de todos los seres vivos.
\item El conjunto de las especies de seres vivos.
\item El alfabeto ingles = $\{a,b,c,d,e,f,g,h,i,j,k,l,m,n,o,p,q,r,s,t,u,v,w,x,y,z \}$
\item El conjunto de los números naturales $\mathbb{N} = \{1,2,3,4,5,...\}$.
\item El conjunto $\mathbb{R}$ de todos los números reales.
\end{itemize}

Aunque los conjuntos pueden ser muy heterogéneos, los conjuntos mas útiles están formados por elementos 
con alguna propiedad en común, algo como


\hfill

$\{ x | P(x) \}$ = El conjunto de los x que tienen la propiedad P(x)

\end{frame}


\begin{frame}{Frame Title}
    \frametitle{Conjuntos}

\textbf{Ejemplos:}

\hfill

\begin{itemize}
\item El conjunto de las canciones de los Beatles = $\{x | \text{x es canción y x fue escrita por los Beatles} \}$
\item El conjunto de los números primos = $\{ n \in \mathbb{N} | n \text{ no es divisible por ningún m con } 1<m<n \}$
\item El conjunto de los números racionales $\mathbb{Q} = \{ r \in R | \exists  m,n \in Z, r=m/n \}$
\item El conjunto de los números irracionales $\mathbb{I} = \{ r \in R | \text{ NO } \exists m,n \in N, r=m/n \}$
\end{itemize}

\end{frame}

\begin{frame}{Frame Title}
    \frametitle{Conjuntos}

Un \textbf{conjunto vacío} es un conjunto que no tiene elementos, así que todos los conjuntos vacíos
son iguales, lo denotamos por $\{ \}$ o por $\O$.

\hfill

\textbf{Ejemplos.}

\hfill

\begin{itemize}
\item El conjunto de todos los perros voladores = $\O$ = El conjunto de los triángulos con 4 lados.
\item $\O = \{ \} \neq \{\{ \}\} = \{ \O \}$
\end{itemize}

\hfill

Si A y B son conjuntos, decimos que A esta \textbf{contenido} en B, o que A es \textbf{subconjunto} de B, si
todos los elementos de A son elementos de B, y escribimos $A \subset B$.
$$
A \subset B \ \forall \ x, x \in A \Rightarrow  x \in B
$$

Por el contrario, cuando A no esta contenido en B escribimos $A \nsubseteq B$

$$
A \nsubseteq B \ \forall \ \exists \ x, x \in A \notin B
$$

\end{frame}


\begin{frame}{Frame Title}
    \frametitle{Conjuntos}

\textbf{Ejemplos.}

\hfill

1. Si A = El conjunto de todas las aves 

   B = El conjunto de todos los animales con alas 

   C = El conjunto de todos los animales que vuelan

Entonces $A \subset B , B \nsubseteq A , B \nsubseteq C , C \subset B , A \nsubseteq C , C 
\nsubseteq A$ .

\hfill

2. Las rectas y planos son conjuntos de puntos. Los puntos son elementos del plano mientras 
que las rectas son subconjuntos del plano.

\end{frame} 


\begin{frame}{Frame Title}
    \frametitle{Conjuntos}

\textbf{Afirmación.} Si A, B y C son conjuntos y $A \subset B \text{ y } B \subset C$ entonces $A \subset 
C$.

\hfill

\textbf{Afirmación.} A = B si y solo si $A \subset B$ y $B \subset A$


\end{frame} 


\begin{frame}{Frame Title}
    \frametitle{Conjuntos}

\textbf{Operaciones con conjuntos.}

\hfill

La \textbf{unión} de dos conjuntos A y B es el conjunto $A \cup B$ formado por los elementos de A y 
los 
elementos de B, es decir
$$
A \cup B = { x | x \in A \& x\in B}
$$

La \textbf{intersección} de dos conjuntos A y B es el conjunto $A \cap B$ formado por los 
elementos 
de A y los 
elementos de B, es decir

$$
A \cap B = \{ x | x \in A \ \& \ x \in B \}
$$

La \textbf{diferencia} entre dos conjuntos A y B es el conjunto A-B formado por los elementos 
de A que no son elementos de B.
$$
A - B = \{ x | x \in A \ y \ x \notin B \}
$$

\end{frame} 


\begin{frame}{Frame Title}
    \frametitle{Conjuntos}

La unión y la intersección son operaciones entre conjuntos, que tienen propiedades parecidas a la
suma y la multiplicación de números naturales.

Lema. Si A, B y C son conjuntos entonces se cumplen: 

\begin{enumerate}
  \item $A \cup B = B \cup A$ La unión es conmutativa
  \item $A \cap B = B \cap A$ La intersección es conmutativa
  \item $(A \cup B) \cup C = A \cup (B \cup C) = A \cup B \cup C$ La unión es asociativa
  \item $(A \cup B) \cap C = A \cap (B \cap C) = A \cap B \cap C$ La intersección es asociativa
  \item $A \cap (B \cup C) = (A \cup B) \cap (A \cup C)$ La intersección se distribuye sobre la unión
  \item $A \cup (B \cap C) = (A \cap B) \cup  (A \cap C)$  La unión se distribuye sobre la intersección
\end{enumerate}

\end{frame}

 
\begin{frame}{Frame Title}
    \frametitle{Conjuntos}

Hay otras maneras de combinar conjuntos para obtener otros conjuntos.

\hfill

El \textbf{producto cartesiano} de dos conjuntos A y B es el conjunto
$A \times B$ cuyos elementos son todas las parejas ordenadas (a,b)
formadas por un elemento de A y un elemento de B:
$$
A \times B = \{ (x,y) | x \in  A , y \in B \}
$$
Ejemplos.

\hfill

\begin{itemize}
   \item Si $A = \{ 1,2,3 \}$ y $B = \{ 1,2 \}$ entonces $A \times B = \{ (1,1), (1,2), (2,1), (2,2), (3,1), (3,2) \}$
   \item Si $A = \{1,2,3 \}$ y $B = \{ \}$ entonces $A \times B = \{ \}$
   \item Si $\mathbb{R}$ es el conjunto de los números reales, entonces $\mathbb{R} \times \mathbb{R}$ es el conjunto de parejas ordenadas de
números reales, también denotado por $\mathbb{R}^2$
\end{itemize}

En el producto cartesiano el orden importa: si $A \neq B$ entonces $A \times B \neq  B \times A$.

\end{frame}

\begin{frame}{Frame Title}
    \frametitle{Conjuntos}

\textbf{Ejercicios.}

\hfill

\begin{enumerate}
   \item Demuestra o da contraejemplos.
   a) $A \cup B = A \cup C \Leftrightarrow  B=C$ b. $A \cap B = A \cap C \Leftrightarrow  B=C$
   \item Demuestra que $\cap$ es conmutativa y asociativa, y que se distribuye sobre $\cup$.
   \item ¿La diferencia de conjuntos es asociativa? Es decir, (A-B)-C = A-(B-C) ?
   \item Muestra que
      a. $(A \cup B)-C = (A-C) \cup (B-C)$ b. $(A \cap B)-C = (A-C) \cap (B-C)$
   \item Encuentra 3 conjuntos A, B y C tales que $A \cap B \neq \O$, $\cap C \neq \O$ y $B \cap C \neq \O$  pero $A \cap B \cap C = \O$.
   \item Si A, B, C son conjuntos, muestra que $(A \times B) \times C \neq A \times (B \times C)$. ¿Como definirías $A \times B \times C$?
\end{enumerate}

\end{frame}

\begin{frame}{Frame Title}
    \frametitle{Conjuntos universales y complementos}

\textbf{Conjuntos universales, complementos.}

\hfill

Si fijamos un conjunto universal U y A está contenido
en U, el \textbf{complemento} de A (en U) es el conjunto $A^c$
formado por los elementos de U que no están en A.

$$
A^c = \{x \in U | x \notin  A \} = U - A
$$

\textbf{Ejemplos.}

\begin{itemize}
   \item Si pensamos en el universo A de todos los animales y el conjunto de los animales que vuelan es V
entonces su complemento $V^c$ es el conjunto de los animales que no vuelan.
   \item La propiedad que define al complemento $A^c$ es la negación de la propiedad que define a A, pero $A^c$
depende de cual sea el universo.
\end{itemize}
\end{frame}

\begin{frame}{Frame Title}
    \frametitle{Conjuntos universales y complementos}

\begin{itemize}
   \item Si $\mathbb{P} = \{ n \in \mathbb{N} | n \text{ es primo } \}$ entonces se entiende que el conjunto 
universal es $\mathbb{N}$ (los números
naturales) y su complemento es $\mathbb{P}^c = \{n \in \mathbb{N} | n \text{no es primo} \}$
%   \item El conjunto de los números racionales es el conjunto
%$ \mathbb{Q} = { x \in \mathbb{R} | \text{ existen enteros m y n tales que x =m/n} \}$
%   \item Su complemento (en el universo de los números reales) es el conjunto de los números irracionales:
%   $\mathbb{I} = \{ x \in \mathbb{R} | \text{ no existen enteros m y n tales que x =m/n} \}$
\end{itemize}

\end{frame}

\begin{frame}{Frame Title}
    \frametitle{Conjuntos universales y complementos}

\textbf{Lema.} Si A y B son conjuntos en un universo U, entonces $A - B = A \cap B^c$

\hfill

\textbf{Lema.} Si A y B son conjuntos en un universo U, entonces $A \subset B$ si y solo si $B^c \subset A^c$


\end{frame}

\begin{frame}{Frame Title}
    \frametitle{Leyes de De Morgan}

\textbf{Leyes de Morgan}

\begin{enumerate}
   \item $(A \cup B)^c = A^c \cap B^c$
   \item $(A \cap B)^c = A^c \cup B^c$
\end{enumerate}


\end{frame} 


\begin{frame}{Frame Title}
    \frametitle{Conjuntos}

\textbf{Ejercicios}

\begin{enumerate}
   \item Si $U= \mathbb{Z}, A= \{n \in \mathbb{Z} | \text{n es múltiplo de 3} \},  B=\{n \in \mathbb{Z} | \text{n es múltiplo de 4} \}$ 
calcula $(A \cup B)^c$ y $(A \cap B)^c$
   \item Demostrar la segunda ley de De Morgan: $(A \cap B)^c = A^c \cup B^c$
   \item Demuestra que el complemento de $(A \cup B) \cap C$ es $(A^c \cap B^c) \cup C^c$.
   \item Si $A \subset U$ y $B \subset V$, calcula el complemento de $A \times B$ en $U \times V$ en términos de los complementos de
A y B en U y V respectivamente (la respuesta es corta pero no es trivial)
\end{enumerate}

\end{frame} 


\begin{frame}{Frame Title}
    \frametitle{Conjuntos de conjuntos}

\textbf{Conjuntos de conjuntos.}

Podemos considerar a conjuntos cuyos elementos sean otros conjuntos.

\hfill

\textbf{Ejemplos.}

\hfill

\begin{itemize}
   \item El conjunto formado por los grupos en una escuela.
   \item El conjunto de todos los intervalos en la recta.
   \item El conjunto de todas las rectas en el plano (este conjunto no es el plano, que es un conjunto de
puntos, sino un conjunto cuyos elementos son las rectas)
\end{itemize}

Si A es cualquier conjunto, el \textbf{conjunto potencia} de A, es el conjunto cuyos elementos son todos
los subconjuntos de A, y es denotado por $2^A$.

\end{frame}

\begin{frame}{Frame Title}
    \frametitle{Conjuntos}

\textbf{Ejemplos.}

\hfill

\begin{enumerate}
   \item Si $A = \{ 1,2,3 \}$ entonces $2^A = \{ \O, \{ 1 \}, \{ 2 \}, \{ 3 \}, \{ 1,2 \}, \{ 1,3 \}, \{ 2,3 \}, \{ 1,2,3 \} \}$
   \item  Si $A = \O$ entonces el único subconjunto de A es $\O$ así que $2^{\O} = { \O }$
\end{enumerate}

\end{frame} 



\end{document}

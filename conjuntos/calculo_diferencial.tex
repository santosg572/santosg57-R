\documentclass{beamer}
\usetheme{Madrid} % Optional theme

\title{Cálculo Diferencial}
\author{L. González-Santos}
\date{\today}

\begin{document}

% Title Page
\begin{frame}
    \titlepage
\end{frame}

% Table of Contents
%\begin{frame}{Outline}
%    \tableofcontents
%\end{frame}

% Example Content Frame
\section{Introduction}

\begin{frame}{Frame Title}
    \frametitle{Conjuntos}

\url{https://es.wikipedia.org/wiki/C\%C3\%A1lculo_diferencial}

\hfill

El cálculo diferencial es una parte del cálculo infinitesimal y del análisis matemático que estudia cómo
cambian las funciones continuas según sus variables cambian de estado. El principal objeto de estudio
en el cálculo diferencial es la derivada. Una noción estrechamente relacionada es la de diferencial de una función. 

\end{frame}

\begin{frame}{Frame Title}
    \frametitle{Conjuntos}

El estudio del cambio de una función es de especial interés para el cálculo diferencial, en concreto el 
caso en el que el cambio de las variables es infinitesimal, esto es, cuando dicho cambio tiende a cero 
(se hace tan pequeño como se desee). Y es que el cálculo diferencial se apoya constantemente en el concepto 
básico del límite. El paso al límite es la principal herramienta que permite desarrollar la teoría del 
cálculo diferencial y la que lo diferencia claramente del álgebra. Desde el punto de vista filosófico 
de las funciones y la geometría, la derivada de una función en un cierto punto es una medida de la 
tasa en la cual una función cambia conforme un argumento se modifica. Esto es, una derivada involucra, 
en términos matemáticos, una tasa de cambio. Una derivada es el cálculo de las pendientes instantáneas de $f( x )$ 
 en cada punto $x$. 


\end{frame} 

\begin{frame}{Frame Title}
    \frametitle{Conjuntos}

Esto se corresponde a las pendientes de las tangentes de la gráfica de dicha función en sus
puntos (una tangente por punto); Las derivadas pueden ser utilizadas para conocer la concavidad de una
función, sus intervalos de crecimiento, sus máximos y mínimos. La inversa de una derivada se
llama primitiva, antiderivada o integral.
\end{frame}

\begin{frame}{Frame Title}
    \frametitle{Diferenciación y diferenciabilidad}

\textbf{Diferenciación y diferenciabilidad}

Una función de una variable es diferenciable en punto $x$ si su derivada existe en ese punto; una función 
es diferenciable en un intervalo si lo es en cada punto $x$  perteneciente al intervalo. 
Si una función no es continua en c, entonces no puede ser diferenciable en c; sin embargo, 
aunque una función sea continua en c, puede no ser diferenciable. Es decir, toda función 
diferenciable en un punto c es continua en c, pero no toda función continua en c es diferenciable en c 
(como $f(x) = |x|$ es continua, pero no diferenciable en x = 0). 


\end{frame}


\begin{frame}{Frame Title}
    \frametitle{Noción de derivada}

Las derivadas se definen tomando el límite de la pendiente de las rectas secantes conforme se van aproximando a la recta tangente. Es difícil hallar directamente la pendiente de la recta tangente de una función porque sólo se conoce un punto de esta, el punto donde ha de ser tangente a la función. Por ello, se aproxima la recta tangente por rectas secantes. Cuando se tome el límite de las pendientes de las secantes próximas, se obtendrá la pendiente de la recta tangente. 


\end{frame}

\begin{frame}{Frame Title}
    \frametitle{Conjuntos}

Para obtener estas pendientes, tómese un número arbitrariamente pequeño que se denominará h. 
h representa una pequeña variación en x, y puede ser tanto positivo como negativo. 
La pendiente de la recta entre los puntos $( x , f ( x ) )$ y $( x + h , f ( x + h ) )$  es

$$
\frac{(x+h)-f(x)}{h}
$$

Esta expresión es un cociente diferencial de Newton. La derivada de f en x es el límite del valor del cociente diferencial conforme las líneas secantes se acercan más a la tangente:

$$
f'(x)=\lim_{h \to 0} \frac{f(x+h)-f(x)}{h}
$$

Si la derivada de f existe en cada punto x, es posible entonces definir la derivada de f como la función 
cuyo valor en el punto x es la derivada de f en x. 

\end{frame}

\begin{frame}{Frame Title}
    \frametitle{Conjuntos}

Puesto que la inmediata sustitución de h por 0 da como resultado una división por cero, calcular la derivada directamente puede ser poco intuitivo. Una técnica consiste en simplificar el numerador de modo que la h del denominador pueda cancelarse. Esto resulta muy sencillo con funciones polinómicas, pero para la mayoría de las funciones resulta demasiado complicado. Afortunadamente, hay reglas generales que facilitan la diferenciación de la mayoría de las funciones. 

\end{frame}

\begin{frame}{Frame Title}
    \frametitle{El cociente diferencial alternativo}

\textbf{El cociente diferencial alternativo}

La derivada de f(x) (tal como la definió Newton) se describió como el límite, conforme h se aproxima a cero. 
Una explicación alternativa de la derivada puede interpretarse a partir del cociente de Newton. 
Si se utiliza la fórmula anterior, la derivada en c es igual al límite conforme h se aproxima a cero de [f(c + h) - f(c)] / h. 
Si se deja que h = x - c (por ende, c + h = x), entonces x se aproxima a c (conforme h tiende a cero). 
Así, la derivada es igual al límite conforme x se aproxima a c, de [f(x) - f(c)] / (x - c). 
Esta definición se utiliza para una demostración parcial de la regla de la cadena. 


\end{frame} 

\begin{frame}{Frame Title}
    \frametitle{Conjuntos}


\end{frame} 

\begin{frame}{Frame Title}
    \frametitle{Conjuntos}


\end{frame} 

\begin{frame}{Frame Title}
    \frametitle{Conjuntos}


\end{frame} 
\end{document}

\begin{frame}{Frame Title}
    \frametitle{Funciones de varias variables}

\textbf{Funciones de varias variables}

Para funciones de varias variables $f :\mathbb{R}^m \to \mathbb{R}^n$, las condiciones de diferenciabilidad son 
más estrictas y requieren más condiciones aparte de la existencia de derivadas parciales. 
En concreto, se requiere la existencia de una aproximación lineal a la función en el entorno de un punto. 
Dada una base vectorial, esta aproximación lineal viene dada por la matriz jacobiana:

$$
lim_{‖ h ‖ \to 0} f(x_0 + h )- f ( x_0 ) − J x_0)) h}{ ‖ h ‖} = 0 
$$

\end{frame} 



\end{document}

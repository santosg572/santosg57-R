\begin{frame}{Frame Title}
    \frametitle{Diferenciación y diferenciabilidad}

\textbf{Diferenciación y diferenciabilidad}

Una función de una variable es diferenciable en punto $x$ si su derivada existe en ese punto; una función 
es diferenciable en un intervalo si lo es en cada punto $x$  perteneciente al intervalo. 
Si una función no es continua en c, entonces no puede ser diferenciable en c; sin embargo, 
aunque una función sea continua en c, puede no ser diferenciable. Es decir, toda función 
diferenciable en un punto c es continua en c, pero no toda función continua en c es diferenciable en c 
(como $f(x) = |x|$ es continua, pero no diferenciable en x = 0). 


\end{frame}

